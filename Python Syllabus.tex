%%%%%%%%%%%%%%%%%%%%%%%%%%%%%%%%%%%%%%%%%
% Short Sectioned Assignment
% LaTeX Template
% Version 1.0 (5/5/12)
%
% This template has been downloaded from:
% http://www.LaTeXTemplates.com
%
% Original author:
% Frits Wenneker (http://www.howtotex.com)
%
% License:
% CC BY-NC-SA 3.0 (http://creativecommons.org/licenses/by-nc-sa/3.0/)
%
%%%%%%%%%%%%%%%%%%%%%%%%%%%%%%%%%%%%%%%%%

%----------------------------------------------------------------------------------------
%	PACKAGES AND OTHER DOCUMENT CONFIGURATIONS
%----------------------------------------------------------------------------------------

\documentclass[paper=a4, fontsize=11pt]{scrartcl} % A4 paper and 11pt font size

\usepackage[T1]{fontenc} % Use 8-bit encoding that has 256 glyphs
\usepackage{fourier} % Use the Adobe Utopia font for the document - comment this line to return to the LaTeX default
\usepackage[english]{babel} % English language/hyphenation
\usepackage{amsmath,amsfonts,amsthm} % Math packages

\usepackage{lipsum} % Used for inserting dummy 'Lorem ipsum' text into the template

\usepackage{sectsty} % Allows customizing section commands
\allsectionsfont{\centering \normalfont\scshape} % Make all sections centered, the default font and small caps

\usepackage{fancyhdr} % Custom headers and footers
\pagestyle{fancyplain} % Makes all pages in the document conform to the custom headers and footers
\fancyhead{} % No page header - if you want one, create it in the same way as the footers below
\fancyfoot[L]{} % Empty left footer
\fancyfoot[C]{} % Empty center footer
\fancyfoot[R]{\thepage} % Page numbering for right footer
\renewcommand{\headrulewidth}{0pt} % Remove header underlines
\renewcommand{\footrulewidth}{0pt} % Remove footer underlines
\setlength{\headheight}{13.6pt} % Customize the height of the header

\numberwithin{equation}{section} % Number equations within sections (i.e. 1.1, 1.2, 2.1, 2.2 instead of 1, 2, 3, 4)
\numberwithin{figure}{section} % Number figures within sections (i.e. 1.1, 1.2, 2.1, 2.2 instead of 1, 2, 3, 4)
\numberwithin{table}{section} % Number tables within sections (i.e. 1.1, 1.2, 2.1, 2.2 instead of 1, 2, 3, 4)

\setlength\parindent{0pt} % Removes all indentation from paragraphs - comment this line for an assignment with lots of text

%----------------------------------------------------------------------------------------
%	TITLE SECTION
%----------------------------------------------------------------------------------------

\newcommand{\horrule}[1]{\rule{\linewidth}{#1}} % Create horizontal rule command with 1 argument of height

\title{	
\normalfont \normalsize 
\textsc{How to Learn to Code} \\ [25pt] % Your university, school and/or department name(s)
\horrule{0.5pt} \\[0.4cm] % Thin top horizontal rule
\huge Python Syllabus \\ % The assignment title
\horrule{2pt} \\[0.5cm] % Thick bottom horizontal rule
}

\author{Amy Pomeroy} % ADD YOUR NAME HERE IF YOU CONTRIBUTE!! 

\date{\normalsize\today} % Today's date or a custom date

\begin{document}

\maketitle % Print the title

%----------------------------------------------------------------------------------------
%	FIRST CLASS
%----------------------------------------------------------------------------------------

\section{First Class - Introduction to Unix and Python} 

%------------------------------------------------

\subsection{Class expectations}

\begin{enumerate}
\item Fill in expectations 
\end{enumerate}

%----------------------------------------------------------------------------------------
%	SECOND CLASS
%----------------------------------------------------------------------------------------

\section{Second Class - Scripting}

%------------------------------------------------

\subsection{Class expectations}

\begin{enumerate}
\item Understand what a script is and what it can do
\item Run a script from the terminal

\end{enumerate}

Includes a PowerPoint to go through with students and a script to
run. 

 %----------------------------------------------------------------------------------------
%	THIRD CLASS
%----------------------------------------------------------------------------------------

\section{Third Class - Jupyter notebooks and strings}


%------------------------------------------------

\subsection{Class expectations}

\begin{enumerate}
\item Fill in expectations
\end{enumerate}

%----------------------------------------------------------------------------------------
%	FOURTH CLASS
%----------------------------------------------------------------------------------------

\section{Fourth Class - File input/output}

%------------------------------------------------

\subsection{Class expectations}

\begin{enumerate}
\item Fill in expectations
\end{enumerate}

%----------------------------------------------------------------------------------------
%	FIFTH CLASS
%----------------------------------------------------------------------------------------

\section{Fifth Class - For loops}

%------------------------------------------------

\subsection{Class expectations}

\begin{enumerate}
\item Fill in expectations 
\end{enumerate}

%----------------------------------------------------------------------------------------
%	SIXTH CLASS
%----------------------------------------------------------------------------------------

\section{Sixth Class - Conditionals}

%------------------------------------------------

\subsection{Class expectations}

\begin{enumerate}
\item Fill in expectations
\end{enumerate}

%----------------------------------------------------------------------------------------
%	SEVENTH CLASS
%----------------------------------------------------------------------------------------

\section{Seventh Class - Functions}

%------------------------------------------------

\subsection{Class expectations}

\begin{enumerate}
\item Fill in expectations
\end{enumerate}

%----------------------------------------------------------------------------------------
%	EIGHTH CLASS
%----------------------------------------------------------------------------------------

\section{Eighth Class - Review of previous materials using numeric data}

%------------------------------------------------

\subsection{Class expectations}

\begin{enumerate}
\item Fill in expectations
\end{enumerate}

%----------------------------------------------------------------------------------------
%	NINETH CLASS
%----------------------------------------------------------------------------------------

\section{Ninth Class - Plotting and Modules}

%------------------------------------------------

\subsection{Class expectations}

\begin{enumerate}
\item Fill in expectations
\end{enumerate}

%----------------------------------------------------------------------------------------
%	TENTH, ELEVENTH, and TWELVTH CLASSES
%----------------------------------------------------------------------------------------

\section{Tenth, Eleventh, and Twelvth Classes - Final project}

%------------------------------------------------

\subsection{Project expectations}

\begin{enumerate}
\item Fill in expectations
\end{enumerate}

%----------------------------------------------------------------------------------------

\end{document}